% Template for Elsevier CRC journal article
% version 1.2 dated 09 May 2011

% This file (c) 2009-2011 Elsevier Ltd.  Modifications may be freely made,
% provided the edited file is saved under a different name

% This file contains modifications for Procedia Computer Science

% Changes since version 1.1
% - added "procedia" option compliant with ecrc.sty version 1.2a
%   (makes the layout approximately the same as the Word CRC template)
% - added example for generating copyright line in abstract

%-----------------------------------------------------------------------------------

%% This template uses the elsarticle.cls document class and the extension package ecrc.sty
%% For full documentation on usage of elsarticle.cls, consult the documentation "elsdoc.pdf"
%% Further resources available at http://www.elsevier.com/latex

%-----------------------------------------------------------------------------------

%%%%%%%%%%%%%%%%%%%%%%%%%%%%%%%%%%%%%%%%%%%%%%%%%%%%%%%%%%%%%%
%%%%%%%%%%%%%%%%%%%%%%%%%%%%%%%%%%%%%%%%%%%%%%%%%%%%%%%%%%%%%%
%%                                                          %%
%% Important note on usage                                  %%
%% -----------------------                                  %%
%% This file should normally be compiled with PDFLaTeX      %%
%% Using standard LaTeX should work but may produce clashes %%
%%                                                          %%
%%%%%%%%%%%%%%%%%%%%%%%%%%%%%%%%%%%%%%%%%%%%%%%%%%%%%%%%%%%%%%
%%%%%%%%%%%%%%%%%%%%%%%%%%%%%%%%%%%%%%%%%%%%%%%%%%%%%%%%%%%%%%

%% The '3p' and 'times' class options of elsarticle are used for Elsevier CRC
%% The 'procedia' option causes ecrc to approximate to the Word template
\documentclass[3p,times,procedia]{elsarticle}
\flushbottom

%% The `ecrc' package must be called to make the CRC functionality available
\usepackage{ecrc}
\usepackage[bookmarks=false]{hyperref}
    \hypersetup{colorlinks,
      linkcolor=blue,
      citecolor=blue,
      urlcolor=blue}
%\usepackage{amsmath}


%% The ecrc package defines commands needed for running heads and logos.
%% For running heads, you can set the journal name, the volume, the starting page and the authors

%% set the volume if you know. Otherwise `00'
\volume{00}

%% set the starting page if not 1
\firstpage{1}

%% Give the name of the journal
\journalname{Procedia Computer Science}

%% Give the author list to appear in the running head
%% Example \runauth{C.V. Radhakrishnan et al.}
\runauth{Grigoryev Mikhail}

%% The choice of journal logo is determined by the \jid and \jnltitlelogo commands.
%% A user-supplied logo with the name <\jid>logo.pdf will be inserted if present.
%% e.g. if \jid{yspmi} the system will look for a file yspmilogo.pdf
%% Otherwise the content of \jnltitlelogo will be set between horizontal lines as a default logo

%% Give the abbreviation of the Journal.
\jid{procs}

%% Give a short journal name for the dummy logo (if needed)
%\jnltitlelogo{Computer Science}

%% Hereafter the template follows `elsarticle'.
%% For more details see the existing template files elsarticle-template-harv.tex and elsarticle-template-num.tex.

%% Elsevier CRC generally uses a numbered reference style
%% For this, the conventions of elsarticle-template-num.tex should be followed (included below)
%% If using BibTeX, use the style file elsarticle-num.bst

%% End of ecrc-specific commands
%%%%%%%%%%%%%%%%%%%%%%%%%%%%%%%%%%%%%%%%%%%%%%%%%%%%%%%%%%%%%%%%%%%%%%%%%%

%% The amssymb package provides various useful mathematical symbols

\usepackage{amssymb}
%% The amsthm package provides extended theorem environments
%% \usepackage{amsthm}

%% The lineno packages adds line numbers. Start line numbering with
%% \begin{linenumbers}, end it with \end{linenumbers}. Or switch it on
%% for the whole article with \linenumbers after \end{frontmatter}.
%% \usepackage{lineno}

%% natbib.sty is loaded by default. However, natbib options can be
%% provided with \biboptions{...} command. Following options are
%% valid:

%%   round  -  round parentheses are used (default)
%%   square -  square brackets are used   [option]
%%   curly  -  curly braces are used      {option}
%%   angle  -  angle brackets are used    <option>
%%   semicolon  -  multiple citations separated by semi-colon
%%   colon  - same as semicolon, an earlier confusion
%%   comma  -  separated by comma
%%   numbers-  selects numerical citations
%%   super  -  numerical citations as superscripts
%%   sort   -  sorts multiple citations according to order in ref. list
%%   sort&compress   -  like sort, but also compresses numerical citations
%%   compress - compresses without sorting
%%
%% \biboptions{authoryear}

% \biboptions{}

% if you have landscape tables
\usepackage[figuresright]{rotating}
%\usepackage{harvard}
% put your own definitions here:x
%   \newcommand{\cZ}{\cal{Z}}
%   \newtheorem{def}{Definition}[section]
%   ...

% add words to TeX's hyphenation exception list
%\hyphenation{author another created financial paper re-commend-ed Post-Script}

% declarations for front matter


\begin{document}
\begin{frontmatter}

%% Title, authors and addresses

%% use the tnoteref command within \title for footnotes;
%% use the tnotetext command for the associated footnote;
%% use the fnref command within \author or \address for footnotes;
%% use the fntext command for the associated footnote;
%% use the corref command within \author for corresponding author footnotes;
%% use the cortext command for the associated footnote;
%% use the ead command for the email address,
%% and the form \ead[url] for the home page:
%%
%% \title{Title\tnoteref{label1}}
%% \tnotetext[label1]{}
%% \author{Name\corref{cor1}\fnref{label2}}
%% \ead{email address}
%% \ead[url]{home page}
%% \fntext[label2]{}
%% \cortext[cor1]{}
%% \address{Address\fnref{label3}}
%% \fntext[label3]{}

\dochead{11th International Young Scientist Conference on Computational Science}%%%
%% Use \dochead if there is an article header, e.g. \dochead{Short communication}
%% \dochead can also be used to include a conference title, if directed by the editors
%% e.g. \dochead{17th International Conference on Dynamical Processes in Excited States of Solids}

\title{A study of the influence of news reports and other contextual open-source information on the consumer behavior of bank card users}

%% use optional labels to link authors explicitly to addresses:
%% \author[label1,label2]{<author name>}
%% \address[label1]{<address>}
%% \address[label2]{<address>}



\author[a]{Grigoryev Mikhail\corref{cor1}}

\address[a]{ITMO University, Kronverksky Pr. 49, bldg. A, St. Petersburg, 197101, Russia}

\begin{abstract}
%% Text of abstract
Insert here your abstract text.
\end{abstract}

\begin{keyword}
Machine Learning ; Natural Language Processing ; Topic Analysis ; Autoregressive Models ; ARIMAX

%% keywords here, in the form: keyword \sep keyword

%% PACS codes here, in the form: \PACS code \sep code

%% MSC codes here, in the form: \MSC code \sep code
%% or \MSC[2008] code \sep code (2000 is the default)

\end{keyword}
\cortext[cor1]{Corresponding author. Tel.: +7-921-592-4920.}
\end{frontmatter}

%\correspondingauthor[*]{Corresponding author. Tel.: +0-000-000-0000 ; fax: +0-000-000-0000.}
\email{mikegrig@inbox.ru}

%%
%% Start line numbering here if you want
%%
% \linenumbers

%% main text

%\enlargethispage{-7mm}
\section{Main Text}
\label{main}

Here introduce the paper, and put a nomenclature if necessary, in a box with the same font size as the rest of the paper. The paragraphs continue from here and are only separated by headings, subheadings, images and formulae. The section headings are arranged by numbers, bold and 10 pt. Here follows further instructions for authors.

\begin{nomenclature}
\begin{deflist}[A]
\defitem{A}\defterm{radius of}
\defitem{B}\defterm{position of}
\defitem{C}\defterm{further nomenclature continues down the page inside the text box}
\end{deflist}
\end{nomenclature}

\subsection{Structure}

%\begin{table}[h]
%\caption{An example of a table.}
%\begin{tabular*}{\hsize}{@{\extracolsep{\fill}}lll@{}}
%\toprule
%An example of a column heading & Column A ({\it{t}}) & Column B ({\it{t}})\\
%\colrule
%And an entry &   1 &  2\\
%And another entry  & 3 &  4\\
%And another entry &  5 &  6\\
%\botrule
%\end{tabular*}
%\end{table}

%\enlargethispage{12pt}

%\cite{Massimo2011}

%Artwork filenames should comply with the syntax ``aabbbbbb.ccc'', where:
%\begin{itemize}
%\item a = artwork component type
%\item b = manuscript reference code
%\item c = standard file extension

%Component types:
%\item gr = figure
%\item pl = plate
%\item sc = scheme
%\item fx = fixed graphic
%\end{itemize}

%\begin{figure}[t]\vspace*{4pt}

%\centerline{\includegraphics{fx1}\hspace*{5mm}\includegraphics{fx1}}

%\centerline{\includegraphics{gr1}}
%\caption{(a) first picture; (b) second picture.}
%\end{figure}

\appendix
\section{An example appendix}
Authors including an appendix section should do so before References section. Multiple appendices should all have headings in the style used above. They will automatically be ordered A, B, C etc.

\subsection{Example of a sub-heading within an appendix}
There is also the option to include a subheading within the Appendix if you wish.
%% References
%%
%% Following citation commands can be used in the body text:
%% Usage of \cite is as follows:
%%   \cite{key}         ==>>  [#]
%%   \cite[chap. 2]{key} ==>> [#, chap. 2]
%%

%The citation must be used in following style: \cite{article-minimal} \cite{article-full} \cite{article-crossref} \cite{whole-journal}.
%% References with BibTeX database:

%\bibliography{xampl}
%\bibliographystyle{elsarticle-harv}


%% Authors are advised to use a BibTeX database file for their reference list.
%% The provided style file elsarticle-num.bst formats references in the required Procedia style

%% For references without a BibTeX database:

 \begin{thebibliography}{}

%% \bibitem must have the following form:
%%   \bibitem{key}...
%%

\bibitem{Massimo2011}
{F}ilippini, Massimo, and Lester C. Hunt. (2011) ``Energy demand and
energy efficiency in the OECD countries: a stochastic demand frontier
approach." {\it Energy Journal} {\bf 32} (2): 59--80.
\bibitem{Massimo2012}
Filippini, Massimo, and Lester C. Hunt. (2012) ``US residential
energy demand and energy efficiency: A stochastic demand frontier
approach." {\it Energy Economics} {\bf 34} (5): 1484--1491.
\bibitem{Thomas2015} 
Weyman-Jones, Thomas, J\'{u}lia Mendon\c{c}a Boucinha, and Catarina
Feteira In\'{a}cio. (2015) ``Measuring electric energy efficiency in
Portuguese households: a tool for energy policy." {\it Management of Environmental Quality: An International Journal} {\bf 26} (3): 407--422.
\bibitem{} 
Saunders, Harry (2009) ``Theoretical Foundations of the Rebound Effect'', in Joanne Evans and Lester Hunt (eds) {\it International Handbook on the Economics of Energy}, Cheltenham, Edward Elgar
\bibitem{} 
Sorrell, Steve (2009) ``The Rebound Effect: definition and estimation'', in Joanne Evans and Lester Hunt (eds) {\it International Handbook on the Economics of Energy}, Cheltenham, Edward Elgar 
 \end{thebibliography}

\end{document}

%%
%% End of file `procs-template.tex'.
